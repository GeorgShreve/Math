\documentclass{report}
\usepackage{setspace}
\usepackage[utf8]{inputenc}
\usepackage{amssymb,amsthm,amsmath,graphicx}
\usepackage{enumerate}
\usepackage{tikz}
\usepackage{caption}
\usepackage{subcaption}

\usepackage[bottom=0.5cm, right=1.5cm, left=1.5cm, top=1.5cm]{geometry}
\providecommand{\abs}[1]{\lvert#1\rvert}
\providecommand{\norm}[1]{\lVert#1\rVert}
\providecommand{\inn}[1]{\langle#1\rangle}
\providecommand{\Z}{\mathbb{Z}}
\providecommand{\R}{\mathbb{R}}
\providecommand{\C}{\mathbb{C}}
\providecommand{\N}{\mathbb{N}}
\providecommand{\A}{\textbf{A}}

\DeclareMathOperator{\Ima}{Im}
\DeclareMathOperator{\Kera}{Ker}
\DeclareMathOperator{\GL}{GL}
\DeclareMathOperator{\deg}{deg}

\newtheorem{theorem}{Theorem}
\newtheorem{proposition}{Proposition}
\neqwtheorem{lemma}{Lemma}
\newtheorem{defi}{Definition}
\newtheorem{intu}{Intuition}
\newtheorem{rmk}{Remark}
\newtheorem{problem}{Problem}

\begin{document}

  \chapter{General Algebra}
  \section{Rings}
  \subsection{Types of Rings}
  We assume all our rings to have unity.
  \begin{enumerate}
  \item[] \textbf{Domain}: A ring $R$ with non non-trivial zero divisors, in other words $xy=0$ implies $x,y=0$ for all $x,y\in R$.
    \begin{enumerate}
    \item[] Commutative: \textbf{Integral Domain/Entire Ring}
    \end{enumerate}

  \item[] \textbf{Noetherian Ring}: A ring $R$ such that every ascending chain of ideals $I_1\subseteq I_2\subseteq\ldots$ of left or right ideals has a largest element. An equivalent property is that every ideal is finitely generated.
    \begin{enumerate}
    \item Any field is Noetherian, since it only has two ideals, the trivial ideal and itself. In this manner, any ring of polynomials over a field is Noetherian, so that every ideal in a polynomial ring over a field is finitely generated.
    \end{enumerate}

  \item[] \textbf{Division Ring}: A ring $R$ such that for every non-zero element of $R$ has an inverse.
    \begin{enumerate}
    \item[] Commutative: \textbf{Field}
    \end{enumerate}
  \end{enumerate}

  
  All rings henceforth are commutative.
  \begin{enumerate}
  \item[] \textbf{Unique Factorization Domain (UFD)/Factorial Ring}: An integral domain where every non-zero element $x\in R$ can be written as a product of irreducible elements $p_i$ of $R$ and a unit $u$
    \[
      x =  up_1\cdots p_n \text{ with } n\geq 0
    \]
    and this representation is unique in the sense that if $q_1,\ldots,q_m$ are irreducible elements of $R$ and $w$ is a unit such that
    \[
      x = wq_1\cdots q_m \text{ with } m\geq 0
    \]
    then $m=n$ and there exists a bijection $\varphi:\{1,\ldots, n\}\to \{1,\ldots, m\}$ such that $p_i=q_{\varphi(i)} u$ for some unit of $R$.

  \item[] \textbf{Principal Ideal Domain (PID)}: An integral domain where every ideal is principal.
    
  \item[] \textbf{Euclidean Domain}: An integral domain where there exists function $f:R\to \Z_{\geq 0}$ such that for any $a,b\in R$ there exists $q,r\in R$ such that
    \[
      a=qb+r \text{ with } r=0 \text{ or } 0<f(r)\leq f(b).
    \]
    In other words a euclidean domain is a domain where the division theorem holds.
  \end{enumerate}
  
    \subsection{Fields}
    \begin{enumerate}
    \item[] \textbf{Algebraically Closed Field}: A field $k$ where every non-constant polynomial in $k[x_1,\ldots,x_n]$ has a zero (root) in $k$.
    \end{enumerate}

    \subsection{Algebras}
    \begin{defi}
      An \textbf{(associative) algebra over a ring} $R$ is a ring $S$ that is also an $R$-module in the precise sense that the addition operators are equal and scalar multiplication obeys $r\cdot xy=(r\cdot x)y=x(r\cdot y)$.
    \end{defi}
    We note that any \textit{commutative} ring is an algebra over itself.
    
    
  \chapter{Hartshorne Chapter 1}
  \section{Affine Varieties}
  We let $k$ be an algebraically closed field.
  \begin{defi}
    \textbf{Affine n-space} over $k$, denoted $\textbf{A}^n_k$ or $\A^n$, is the set $n$-tuples of elements of $k$. An element of $\A^n$ will be called a \textbf{point}, and its entries \textbf{coordinates}.
  \end{defi}

  Let $A=k[x_1,\ldots, x_n]$ be the ring of polynomials in $n$ variables over $k$. We view these as functions from $\A^n\to k$.

  \begin{defi}
    If $T\subseteq A$ is a set of polynomials we define its zeros as
    \[
      Z(T) = \{P=(a_1,\ldots, a_n)\in \A^k\mid f(P)=0\forall f\in T\}.
    \]
    Recalling that $A$ is Noetherian, if $\mathfrak{a}$ is the ideal generated by $T$ in $A$, clearly $Z(\mathfrak{a})=Z(T)$ and so $T$ is generated by a finite number of polynomials $f_1,\ldots, f_n$. 
  \end{defi}

  Indeed, we define
  \begin{defi}
    A subset $Y\subseteq \A^n$ is an \textbf{Algebraic Set} if there exists $T\subseteq A$ such that $Y=Z(T)$.
  \end{defi}
  Recalling the above we see that a subset of Affine n-space is algebraic if it is the set of zero's of a finite number of polynomials. We induce the Zariski topologie by letting the compliments of the algebraic sets in $A^n$ be the open sets. We recall the definition of a topology:
  \begin{defi}
    A \textbf{topology} is a set $X$ and a set $\Omega$ of subsets of $X$ satisfying
    \begin{enumerate}
    \item $ \emptyset, X\in \Omega$
    \item If $\{U_\alpha\}$ is a family of open subsets of $X$ (i.e. elements of $\Omega$) then $\cup_\alpha U_\alpha$ is an open set of $X$. In other words the open sets are closed under arbitrary unions.
    \item If $\{K_i\}$ is a finite family of open subsets of $X$, then $\cap_i K_i$ is an open set of $X$. In other words the open sets are closed under finite intersections.
    \end{enumerate}
  \end{defi}
  We see that the Zariski topology as defined above is indeed a topology, for
  \begin{proposition}
    \begin{enumerate}
    \item The union of two algebraic sets is algebraic.
    \item The intersection of arbitrary algebraic sets is algebraic.
    \item The empty set and $\A^n$ are algebraic.
    \end{enumerate}
  \end{proposition}
  \begin{proof}
    \begin{enumerate}
    \item Suppose $X_{1},X_2\subseteq \A^n$ are algebraic sets. Then there exists $T_1T_2\subseteq A$ such that $X_1=Z(T_1)$ and $X_2=Z(T_2)$ respectively, so that any $x\in X_1\cup X_2$ is a zero for every polynomial in either $T_1$ or $T_2$, but then in particular its a zero for all polynomial in $T_1T_2$. Thus $X_1\cup X_2\subseteq Z(T_1T_2)$. Conversely suppose $x\in Z(T_{1}T_{2})$, and without loss of generality suppose $x\neq X_{1}$. Then for some $f\in T_{1}$, $f(x)\neq 0$ so that since $(fg)(x)=0$ for all $g\in T_{2}$, $g(x)=0$ for all $g\in T_{2}$ and so $x\in X_2$. We are done.

      \item Suppose 

  
  \section{Interlude - Nullstellensatz}
  From https://stanford.edu/~sfh/nullstellensatz.pdf
  I believe we are assuming every ring here to commutative.
  \subsection{Dimension Theory}
  We begin with definitions for various notions of ring extensions being \textit{finite}. Let $R,S$ be rings.
  \begin{defi}
    A ring $S$ is said to be \textbf{finite over} $R$ if it is finitely generated as a module over $R$. In other words there exists a morphism $f:R\to S$ that allows $S$ to be viewed as an $R$ module (by closure), and to establish the finite generation there is an $R$-linear surjection $R^n\to S$ for some $n$. If $f$ is injective, then we say $S$ is a \textbf{finite extension} of $R$.
  \end{defi}
  In this sense a ring $S$ being finite over a ring $R$ means it arises as a finitely generated $R$-module for some given morphism $f:\R\to S$ that etsablishes the action of its ring of scalars. For $s\in S$ and $r\in R$ we view $r\cdot s=f(r)s$ (if we consider it as a left $R$-module).
  \begin{defi}
    A ring $S$ is said to be \textbf{finite-type over} $R$ if it is finitely generated as an algebra over $R$. That is there exists a ring ismorphism $f:\R\to S$ that allows $S$ to be viewed as an $R$ module, and surjective $R$-linear ring homomorphism (i.e. a morphism of algebras) $R[x_1,\ldots,x_n]\to S$. Equivalently, $S$ is isomorphic to a quotient of the $R$-algebra $R[x_1,\ldots,x_n]$ for some $n$.
  \end{defi}
  We see that being finite type over $R$ is a weaker condition than being finite over $R$. We have a final notion
  \begin{defi}
    Let $f:R\to S$ be a ring hom. $s\in S$ is \textbf{integral over} $R$ if it is the root of a monic polynomial in $f(R)[x]$ (a single variable polynomial with a trivial leading coefficient). We say that $S$ is \textbf{integral over} $R$ if every element of $S$ is integral over $R$, in which case $f$ is a \textbf{integral homomorphism}. If $f$ is injective, it is an \textbf{integral extension}.
  \end{defi}

  The following proposition connects these notions
  \begin{proposition}
    A ring $S$ is finite over $R$ \iff $S$ is of finite-type and integral over $R$.
  \end{proposition}
  We begin with a lemma,
  \begin{lemma}
    Let $f:R\to S$ be a ring hom. Then $s\in S$ is integral over $R$ \iff it is contained in an $R$-subalgebra that is finite over $R$.
  \end{lemma}
  \begin{proof}
    $(\Rightarrow)$ Suppose $s\in S$ is integral over $R$. Then there is a monic polynomial $p\in f(R)[x]$ that has $s$ as a zero. In particular this polynomial has scalars in $R$ (since we consider scalar multiplication $r\cdot x=f(r)x$). Then the module generated by $1,s,\ldots, s^{\deg p-1}$ is closed under mulitplication, because $s^{\deg p}$ can be expressed as a sum of lesser terms of $p(s)$. Then by definition this finitely generated $R$-submodule (of $S$) is infact an $R$-subalgebra that is finite over $R$.
    \begin{rmk}
      We must note that the module being closed under multplication is not the same as the generated also generated some subset of $R$, which it always does. Being closed under multiplication in this sense means that any product of elements of the module is expressable as a sum of scaled elements of the module.
    \end{rmk}
    
    
  



 \end{document}